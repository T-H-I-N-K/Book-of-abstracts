\providecommand{\main}{..} 
\documentclass[\main/boa.tex]{subfiles}

\begin{document}

\section{M-quantile regression in R}

\begin{center}
  {\bf 
  \index[a]{Beręsewicz Maciej} Maciej Beręsewicz$^{1^\star}$
  \index[a]{Salvati Nicola} Nicola Salvati$^{2}$, 
  \index[a]{Marchetti Stefano} Stefano Marchetti$^{2}$}
\end{center}

\vskip 0.3cm

\begin{affiliations}
\begin{enumerate}
\begin{minipage}{0.915\textwidth}
\centering
\item Poznan University of Economics and Business, Poland \\[-2pt]
\item University of Pisa, Italy
\end{minipage}
\end{enumerate}
$^\star$Contact author: \href{mailto:maciej.beresewicz@ue.poznan.pl}{\nolinkurl{maciej.beresewicz@ue.poznan.pl}}\\
\end{affiliations}

\vskip 0.5cm

\begin{minipage}{0.915\textwidth}
\keywords m-quantile regression, small area estimation, robust estimation
\end{minipage}

\vskip 0.8cm

In the presentation we will discuss M-quantile regression as an alternative for robust linear mixed models and quantile mixed models. First, we discuss theory and current research on M-quantile models. We give examples of its application, in particular for small area estimation. Second, we provide a brief review of available R routines which were prepared mainly for research projects (e.g. SAMPLE) or scientific papers. Finally, we give an overview of on-going work on an R package for M-quantile regression.

\end{document}

